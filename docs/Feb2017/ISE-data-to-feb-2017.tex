\documentclass[11pt,twoside,a4paper]{article}

\usepackage{graphicx}
\usepackage{xcolor}
\usepackage{hyperref}
\usepackage{fullpage}
\usepackage{tikz}
\usetikzlibrary{arrows.meta,shapes,backgrounds,positioning,shapes.multipart}
\usepackage{textpos} 
\hypersetup{% See http://www.tug.org/applications/hyperref/manual.html#x1-120003.8
  colorlinks=true,
  linkcolor=green,
  urlcolor=blue,
  linkbordercolor=red,
}
\title{Linked Open Data for the Solidarity Economy}
\author{Matt Wallis, Institute for Solidarity Economics}
%\date{\today}
\date{February 23, 2017}
\begin{document}
\maketitle

The Institute for Solidarity Economics has been working on Linked Open Data for the Solidarity Economy,
producing an experimental data-set of UK co-ops, and an application to view the co-ops on a geographical map.
The map also provides as way to browse educational material about the Linked Open Data.

This document describes the work to date.
It is based on a presentation given at the Open:2017,
Platform Co-operatives (\url{https://2017.open.coop/}) conference in London, February 17, 2017.

\section{Institute for Solidarity Economics}
  \begin{center}
    \includegraphics[height=2cm,width=2cm]{ise-logo.jpg}

    \begin{tabular}{l l}
      Website: & \url{http://solidarityeconomics.org} \\
      Twitter: & \href{https://twitter.com/SolidarityEcon}{@SolidarityEcon} \\
    \end{tabular}
  \end{center}

Our aim is to support the Solidarity Economy movement through education, research, and finding opportunities for collaboration.
What we mean by the Solidarity Economy is a grassroots movement that is building a fair and ecological alternative to Capitalism.


\subsection{Why data?}

Data fuels the World Wide Web -- they are many examples of the power of data. 

One of the goals of the Institute for Solidarity Economics is to make it easier to discover the initiatives that are part of the Solidarity Economy,
for example by putting those initiatives on a geographic map.
A geographic map provides a view of the data. Other simple views (like an alphabetic directory) also have their uses.
In any case, the data comes before any useful view of the data.

We can also see a need for applications that do more than provide a view of SE initiatives -- 
applications that help the initiative to operate;
applications that help initiatives to co-operate.
These applications will also depend on data.
Interoperability between applications is greatly improved by interoperability between the data used by those applications.
It is because of this fundamental dependency on data that ISE decided to take a closer look at it.


\subsection{Social considerations}

Data ownership is power. The social issues around this power need careful consideration.

\subsubsection{Ownership of data}
By publishing data about something, you are making claims about that thing, and the actual 'owners' of that thing may disagree with the claims you are making.
This issue can be avoided if the 'owners' of the thing are able to publish their own data.
So it is important that any technology used for publishing and accessing data allows the actual owner to choose to publish their own data.

We are all familiar with the power wielded by the big corporations who dominate data.
We want a solution that permits the data to be owned by the people whose resources are described by that data.
They may choose to allow someone else to own it, but we want to give them that choice.

\subsubsection{Truth -- who's making the claims}
Because the publisher of data is the one who is making a claim about the things described by their data,
it is important that the technology chosen to publish data allows users of that data to see who is publishing the data, who is making these claims.
For example, if ISE publishes data about initiatives within the Solidarity Economy, 
then ISE may be claiming that each of these initiatives satisfies ISE's view of what it means to be in the Solidarity Economy.
Other organizations may have a slightly different view of which initiatives should be regards as \textit{in} or \textit{out} of the Solidarity Economy.
So, it's important that it is possible to identify \textit{who} is making the claims expressed in the data.

\subsubsection{Prefer links to copies}
It is usually better to refer to someone else's data rather than to create your own version (a copy) of that data.
If there are two versions of the same data, then any changes should be made to both, and this rarely happens in practice.
The original owners of the data may be misrepresented by a copy of their data which does not contain the changes that they have made since the copy was made.
Similarly, they may be misrepresented by changes made to the copy.
So, we prefer links to data, rather than copies.

\subsubsection{Trust and provenance}
How can we trust the data we use? How can we believe it to be true?
Usually we depend on trusting the \textit{source} of information in order to believe in it.
So, it's vital that we can determine the sources of the information,
its \textit{provenance},
particularly when we are looking at situations where data may have been gathered from many different sources.

To summarize:
\begin{itemize}
  \item Permit data to be distributed rather than centralized.
  \item Allow data to be owned by the owner of the resource which it describes.
  \item References to other people's data are better than copies of it.
  \item The provenance of data should be traceable.  
\end{itemize}


\section{What we did}
The following engineering principles guided the path we took:
  \begin{itemize}
    \item Use existing best practice
    \item Re-use existing software
    \item Re-use existing data (link to it, don't copy it!)
    \item Re-use existing data models
    \item Open source
      \begin{itemize}
	\item Use open source software
	\item ISE publishes work with open source licenses
      \end{itemize}
  \end{itemize}

\subsection{Research and Development}
We decided that Linked Open Data (LOD) was a good candidate for best practice in this field.
By Linked Open Data, we mean the set of technologies and specifications \href{https://www.w3.org/standards/semanticweb/data}{described by W3C}, based on an original idea by Tim Berners-Lee to extend the World Wide Web (the ``Web of Documents'') to include a ``Web of Data''. 

In order to understand LOD properly, we wanted to use it in a realistic situation: 
To create and deploy a LOD dataset of a good size, the data describing initiatives within the Solidarity Economy,
and to create a simple map application to display the initiatives described in the dataset. 
This would exercise the creation, deployment and consumption of the data.

The development work undertaken by ISE is available open source on GitHub:
\url{https://github.com/p6data-coop/ise-linked-open-data}.

\subsection{Co-ops UK data}

Starting in 2015, Co-ops UK have been
\href{https://www.uk.coop/economy2015/access-data}{publishing open data}
about co-ops.
This dataset provides information about over 10,000 co-ops and co-op outlets.
It is available in Comma-Separated Value (CSV) format.
CSV is a machine-readable format, which removes one of the obstacles to converting it to LOD.

\subsection{ESSGLOBAL}

To model the Co-ops UK as Linked Data,
we wanted to re-use existing data models, if possible.
To this end, we found the ESSGLOBAL Metadata Application Profile
(also known the Dublin Core Application Profile for the Social and Solidarity Economy - DCAP-SSE).
This work was undertaken by RIPESS Europe and is described in 
\href{http://ripess.eu/wp-content/uploads/2014/07/ESSglobal_interop_guidelines.pdf}{this document}.

During our work, we made a few corrections and updates to ESSGLOBAL.
This will form the basis of new releases of ESSGLOBAL, and is stored in a GitHub repository:
\url{https://github.com/essglobal-linked-open-data/map-sse}.

\subsection{LOD - Links}

\subsubsection{Links to Ordnance Survey}

We wanted to put the co-ops from the Co-ops UK dataset onto a map.
However, the dataset does not provide the geographic location (e.g. latitude and longitude) of co-ops.
But the dataset does provide postcodes, and Ordnance Survey publish Linked Open Data that provides the latitude and longitude (of the "centre"?) of each postcode.
So this was an opportunity both to obtain the location needed to put things on a map, and also to experiment with links to other dataset from our Linked Open Data.

Of course, using the location of the (centre of) the postcode is not as accurate as using the exact location of the co-op. 
For an average UK postcode, all addresses are with about 200m of the centre.
But postcodes vary greatly in size, much bigger in the countryside than than in built-up areas.
If the exact location of a co-op is know, then this can be added to the Linked Open Data.
The location based on the postcode should only be used when the exact location is unavailable.

See \hyperref[sec:tech:os]{section \ref{sec:tech:os}} for technical details.

\subsubsection{Links to Companies House}

Some UK co-ops are registered with Companies House.
For these co-ops, the Co-ops UK dataset provides the registered number.
Companies House provides Linked Open Data for all companies, and you can make a link to the data for a particular company if you know the registered number of that company.

In the spirit of experimentation, we decided to include these links to Companies House in the Linked Open Data for co-ops,
and then to see how we could use the data held by Companies House that was now reachable through the link.
We found that the LOD from Companies House provides a considerable amount of information including the ``line of business'' 
(actually, the \href{https://en.wikipedia.org/wiki/Standard_Industrial_Classification}{SIC Code})
of each registered organization.

See \hyperref[sec:tech:ch]{section \ref{sec:tech:ch}} for technical details.

\subsubsection{The power of links to other datasets}

By combining the information from Ordnance Survey and Companies House, 
we can get answers to questions such as: 
for a given co-op, what are the other companies in the same line of business as this co-op, which are also geographically close to that co-op?
Of course, there are many other questions that can be answered,
but we concentrated on this one as an illustrative example of the power of linking to other LOD datasets.

See \hyperref[sec:tech:nearby]{section \ref{sec:tech:nearby}} for technical details.

asdas
\subsection{Map of UK co-ops}
\subsection{Linked Open Data}
\subsection{Links}
  %\centering
  \begin{center}
  %\begin{tikzpicture}[scale=.8, show background rectangle, node distance=1cm]
  %\begin{tikzpicture}[remember picture, show background rectangle, node distance=1cm]
  \begin{tikzpicture}[remember picture, node distance=1cm]
    \tikzstyle{every text node part} = [align=center]
    \tikzstyle{obj node} = [ellipse, fill=blue!20]
    \tikzstyle{obj path} = [->, draw=blue!20]
    \tikzstyle{label node} = [draw, text=black]
    \tikzstyle{data node} = [rectangle, fill=green!20]
    \tikzstyle{extdata node} = [rectangle, draw, fill=green!20]
    \tikzstyle{app node} = [rectangle, fill=red!20]
    \tikzstyle{label node} = [midway, auto]
    \tikzstyle{label text} = [align=left]
    \node[obj node] (iseobj) {\underline{Links to other data-sets} e.g. \\ {http://os.gov/postcode/SA335AJ} \\ {http://ch.gov/company/01113761}};
    \node[label text] (datalabel) [left = of iseobj] {Example \\ data:};
    \node[extdata node] (osdata) [below = of iseobj ] {OS data};
    \node[data node] (isedata) [left = of osdata] {Our data};
    \node[extdata node] (chdata) [right = of osdata] {CH data};
    \node[label text] (datasetlabel) at (isedata-|datalabel) {Data-sets:};

    \node[app node] (mapapp) [below left = of osdata] {Map};
    \node[app node] (nearestapp) [below right = of osdata] {Nearby};
    \node[label text] (appslabel) at (mapapp-|datasetlabel) {Apps:};
    \draw[->] (isedata) -- (iseobj) node[label node]{contains};
    \draw[->] (iseobj) -- (osdata) node[midway, left]{link};
    \draw[->] (iseobj) -- (chdata) node[midway, right]{link};
    \draw[<->] (mapapp) -- (isedata) node[midway, right]{query};
    \draw[<->] (mapapp) -- (osdata) node[midway, right]{query};
    \draw[->] (mapapp) -- (nearestapp) node[midway, below]{http://ch.gov/...};
    \draw[<->] (nearestapp) -- (osdata) node[midway, left]{query};
    \draw[<->] (nearestapp) -- (chdata) node[midway, right]{query};
  \end{tikzpicture}
    \begin{itemize}
      \item Existing data: CH = Companies House
      \item Existing data: OS = Ordnance Survey
    \end{itemize}
  \end{center}
\section{Technical Details}
    \subsection{RDF library}
    \subsection{URIs}
    \subsubsection{Real world objects and documents about them}
    TODO: say something about the use of id and doc and content negotiation.
    Maybe refer to \href{https://data.gov.uk/resources/uris}{Creating URIs for UK gov data}
    \subsubsection{Content negotiation}
    \subsubsection{purl}
    TODO: purl.org, Persistence.
    \subsubsection{Example: dereferencing a URI}
    TODO: Walk through purl and content negotiation
    \subsection{Geographic data}
    \label{sec:tech:os}
    \subsection{Links to Companies House}
    \label{sec:tech:ch}
    \subsection{Virtuoso triple store}
    \subsection{SPARQL queries}
    \subsubsection{Querying the co-ops dataset}
    \subsubsection{Finding similar companies nearby}
    \label{sec:tech:nearby}
    \subsection{purl.org}

\section{Conclusions}
\subsection{Links to more on LOD}
    \begin{itemize}
      \item \href{https://www.w3.org/TR/ld-bp/}{W3C: Best Practices for Publishing Linked Data}
      \item \href{https://www.gov.uk/government/uploads/system/uploads/attachment_data/file/60975/designing-URI-sets-uk-public-sector.pdf}{Designing URI Sets for the UK Public Sector}
    \end{itemize}
\end{document}
